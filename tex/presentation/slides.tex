\PassOptionsToPackage{unicode=true}{hyperref} % options for packages loaded elsewhere
\PassOptionsToPackage{hyphens}{url}
\documentclass[ignorenonframetext,]{beamer}
\IfFileExists{pgfpages.sty}{\usepackage{pgfpages}}{}
\setbeamertemplate{caption}[numbered]
\setbeamertemplate{caption label separator}{: }
\setbeamercolor{caption name}{fg=normal text.fg}
\beamertemplatenavigationsymbolsempty

  \usepackage{lmodern}
\usepackage{amssymb,amsmath}
\usepackage{fixltx2e} % provides \textsubscript

\usepackage{fontspec}
\defaultfontfeatures{Ligatures=TeX,Scale=MatchLowercase}








  \usetheme[background=dark]{metropolis}






% use upquote if available, for straight quotes in verbatim environments
\IfFileExists{upquote.sty}{\usepackage{upquote}}{}
% use microtype if available
\IfFileExists{microtype.sty}{%
  \usepackage{microtype}
  \UseMicrotypeSet[protrusion]{basicmath} % disable protrusion for tt fonts
}{}


\newif\ifbibliography
  \usepackage[style=authoryear,]{biblatex}
      \addbibresource{/home/alal/Dropbox/MyLibrary2.bib}
  

\hypersetup{
      pdftitle={Presentation Topic},
        pdfauthor={Apoorva Lal},
          pdfborder={0 0 0},
    breaklinks=true}
%\urlstyle{same}  % Use monospace font for urls





  \usepackage{longtable,booktabs}
  \usepackage{caption}
  % These lines are needed to make table captions work with longtable:
  \makeatletter
  \def\fnum@table{\tablename~\thetable}
  \makeatother

  \usepackage{graphicx,grffile}
  \makeatletter
  \def\maxwidth{\ifdim\Gin@nat@width>\linewidth\linewidth\else\Gin@nat@width\fi}
  \def\maxheight{\ifdim\Gin@nat@height>\textheight0.8\textheight\else\Gin@nat@height\fi}
  \makeatother
  % Scale images if necessary, so that they will not overflow the page
  % margins by default, and it is still possible to overwrite the defaults
  % using explicit options in \includegraphics[width, height, ...]{}
  \setkeys{Gin}{width=\maxwidth,height=\maxheight,keepaspectratio}

% Prevent slide breaks in the middle of a paragraph:
\widowpenalties 1 10000
\raggedbottom

  \AtBeginPart{
    \let\insertpartnumber\relax
    \let\partname\relax
    \frame{\partpage}
  }
  \AtBeginSection{
    \ifbibliography
    \else
      \let\insertsectionnumber\relax
      \let\sectionname\relax
      \frame{\sectionpage}
    \fi
  }
  \AtBeginSubsection{
    \let\insertsubsectionnumber\relax
    \let\subsectionname\relax
    \frame{\subsectionpage}
  }



\setlength{\parindent}{0pt}
\setlength{\parskip}{6pt plus 2pt minus 1pt}
\setlength{\emergencystretch}{3em}  % prevent overfull lines
\providecommand{\tightlist}{%
  \setlength{\itemsep}{0pt}\setlength{\parskip}{0pt}}

  \setcounter{secnumdepth}{0}



% import maths shortcuts
\input{/home/alal/Templates/boilerplate/math_shortcuts.tex}

  \definecolor{mLightBrown}{HTML}{EB811B}
  \definecolor{mLightGreen}{HTML}{14B03D}
  \setbeamercolor{block body}{bg=mDarkTeal!30}
  \setbeamercolor{frametitle}{bg=mLightBrown}

  \title[]{Presentation Topic}


  \author[
        Apoorva Lal
    ]{Apoorva Lal}

  \institute[
    ]{
    Stanford
    }

\date[
      \today
  ]{
      \today
        }


\usepackage{appendixnumberbeamer}
\usepackage{booktabs}
\usepackage[scale=2]{ccicons}
\usepackage{pgfplots}
\usepgfplotslibrary{dateplot}
\usepackage{xspace}

\begin{document}

% Hide progress bar and footline on titlepage
  \begin{frame}[plain]
  \titlepage
  \end{frame}


  \begin{frame}
  \tableofcontents[hideallsubsections]
  \end{frame}

\hypertarget{scratch}{%
\section{Scratch}\label{scratch}}

\begin{frame}{Title here}
\protect\hypertarget{title-here}{}
\parencite{leamer1983let}

\scriptsize cite but don't read
\end{frame}

\begin{frame}{Another slide}
\protect\hypertarget{another-slide}{}
The theme provides sensible defaults to \emph{emphasize} text,
\alert{accent} parts or show \textbf{bold} results.

\[
y = \beta_0 + \vee{x}_i' \ve{\beta} + \epsi_i
\]

Hello, world!
\end{frame}

\begin{frame}{Mathy slide}
\protect\hypertarget{mathy-slide}{}
\metroset{block=fill}
\begin{alertblock}{AIPW Estimator}
$$
\wh{\tau}_{\text{ipw}}^{\text{ate}} = \ooN \sumin \Bigpar{
  \underbrace{\frac{Y_iD_i}{\hat{\pi}(\Vect{X}_i)}}_{\Exp{Y_1}} -
  \underbrace{\frac{Y(1 - D_i)}{(1 - \hat{\pi}(\Vect{X}_i))}}_{\Exp{Y_0}}}
= \ooN \sumin Y_i \Bigpar{\frac{D_i}{\wh{\pi}(\ve{X}_i)} - \frac{1 -
D_i}{1 - \wh{\pi} (\ve{X}_i)}}
$$

\end{alertblock}

Places highest weight for observations with \(\wh{\pi} \approx 0.5\)
\end{frame}

\hypertarget{core-presentation-ingredients}{%
\section{Core Presentation
Ingredients}\label{core-presentation-ingredients}}

\begin{frame}{Figure from R}
\protect\hypertarget{figure-from-r}{}
\includegraphics{slides_files/figure-beamer/figscatter-1.pdf}
\end{frame}

\begin{frame}[fragile]{R Table Example}
\protect\hypertarget{r-table-example}{}
A simple \texttt{knitr::kable} example:

\begin{longtable}[]{@{}lrrrrrrrr@{}}
\caption{(Parts of) the mtcars dataset}\tabularnewline
\toprule
& mpg & cyl & disp & hp & drat & wt & qsec & vs\tabularnewline
\midrule
\endfirsthead
\toprule
& mpg & cyl & disp & hp & drat & wt & qsec & vs\tabularnewline
\midrule
\endhead
Mazda RX4 & 21.0 & 6 & 160 & 110 & 3.90 & 2.620 & 16.46 &
0\tabularnewline
Mazda RX4 Wag & 21.0 & 6 & 160 & 110 & 3.90 & 2.875 & 17.02 &
0\tabularnewline
Datsun 710 & 22.8 & 4 & 108 & 93 & 3.85 & 2.320 & 18.61 &
1\tabularnewline
Hornet 4 Drive & 21.4 & 6 & 258 & 110 & 3.08 & 3.215 & 19.44 &
1\tabularnewline
Hornet Sportabout & 18.7 & 8 & 360 & 175 & 3.15 & 3.440 & 17.02 &
0\tabularnewline
\bottomrule
\end{longtable}
\end{frame}

\begin{frame}{Animation}
\protect\hypertarget{animation}{}
\begin{itemize}[<+->]
\tightlist
\item
  this
\item
  Now this
\item
  And now this
\end{itemize}
\end{frame}

\begin{frame}{Two Columns}
\protect\hypertarget{two-columns}{}
\begin{columns}
\begin{column}{0.5\textwidth}

  \bi
    \item eh?
    \item bleh
  \ei

\end{column}
\begin{column}{0.5\textwidth}
    \begin{center}
     \includegraphics[width=1\textwidth]{../../output/figures/fig.pdf}
     \end{center}
\end{column}
\end{columns}
\end{frame}


  \begin{frame}[allowframebreaks]{}
  \bibliographytrue
  \printbibliography[heading=none]
  \end{frame}


\end{document}
